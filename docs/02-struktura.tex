\section{Struktura a architektura aplikace}

Aplikace striktně odděluje frontendovou a backendovou část. Veškeré PHP skripty
a další datové soubory, jako jsou kaskádové styly, týkající se zobrazování dat
uživateli se nachází v~adresáři \verb|frontend|.

V~kořeni tohoto adresáře se nacházejí soubory, které obstarávají jednotlivé
části stránky, jmenovitě hlavičku, nabídku, prostor na systémové zprávy,
obsah a patičku.

Dále je zde adresář \verb|frontend/action|, kde se nacházejí skripty obsluhující
jednotlivé obrazovky, např. pro vkládání článků, registraci apod.

Jádro aplikace se nachází v~adresáři \verb|core| a obsahuje základní třídy
aplikace. Tyto obstarávají připojení k~databázi, vytváření systémových zpráv,
konfiguraci, přihlašování a další nezbytné části aplikace. Většina těchto tříd
je realizována dle návrhového vzoru \textit{jedináček}.

Podadresář \verb|core/data| obsahuje datové třídy, které kromě přenosu informací
o~článcích, uživatelích, recenzích atd. samy obstarávají vlastní ukládání
a načítání dat z~databáze.

Posledním adresářem je \verb|cmd|, v~němž se nacházejí PHP skripty obsluhující
příkazy od uživatele, jako je registrace, nahrání článků a recenzí, přihlašování
a další úkony.
